% !TEX root = ../Rom-abcd.tex
%% -----------------------------------------------------
%% -- Musterdatei für die Erstellung des Beitrags
%% -- zum Romseminar. Um ein einheitliches Aussehen zu ermöglichen:
%% -- Bitte die Definitionen nicht ändern. 
%% -- Stand: 2024/03/12
%% -----------------------------------------------------

%% -- Definitionen, damit die Eingabe einfacher wird
%% -- Bitte \renewcommand belassen und die entsprechenden Angaben
%% -- eintragen.
%% --
\renewcommand{\LongTitel}{Langform des Titels}
\renewcommand{\ShortTitel}{Kurzform des Titel}
\renewcommand{\AutorenBeitrag}{Autor1, Autor2 \& Autor3}

%% -- Kapitelüberschrift und Eintrag in TOC
%% -- Bitte hier nichts ändern
%% --
\addchap[\ShortTitel]{\LongTitel}
\addtocontents{toc}{\textsc{\AutorenBeitrag}}
\addtocontents{toc}{}

%% -- Kopfzeile 
%% -- Gerade Seiten : Autoren
%% -- Ungerade Seiten: Kurztitel
%% -- Bitte hier nichts ändern
%% --
\markleft{\textsc{\AutorenBeitrag}}	
\markright{\textsc{\ShortTitel}}	

%% -- Titelseite des Beitrags inkl. Bild
%% -- des/der Vortragenden. Bitte autor1 entsprechend anpassen
%% --
\begin{center}
	\textsc{\Large \AutorenBeitrag}\\[1.5em]
	\includegraphics[width=8cm, height=8cm, keepaspectratio=true]{./content-abcd/autor1.jpg}
\end{center}

%% -- Wer einen sinnvollen Spruch nutzen will
%% -- dann bitte mit \dictum[]{}; Größe kann man anpassen
%% -- falls es unschön aussieht
%% --
\renewcommand{\dictumwidth}{0.45\textwidth}
\dictum[\href{https://de.wikipedia.org/wiki/Albert_Einstein}{Albert Einstein}]{Wahnsinn ist, dasselbe immer  \mbox{wieder} zu tun und andere Ergebnisse zu erwarten.}
%%
\begin{quote}
Für eine kurze Zusammenfassung des Beitrags -- sollte schon sein.
\end{quote}

%% -- Referenzen des Beitrags am Ende des Artikels
%% -- und keine Überschneidung mit den Literaturzitaten 
%% -- andere Beiträge
%% --
\begin{refsection}
%% --
\section*{Erster Hauptabschnitt}		%% 
\subsection*{Erster Unterabschnitt}		%% 
\blindtext								
%%				
\subsection*{Zweiter Unterabschnitt}
\blindtext	
%% --
\section*{Zweiter Hauptabschnitt}		%% 
\subsection*{Erster Unterabschnitt}		%% 
\blindtext	
%% -- etc.
%% --

%% -- Literaturverzeichnis
%% --
\RaggedRight
\nocite{*}
\printbibliography
\end{refsection}
%% -- Ende des Beitrags
%% --

